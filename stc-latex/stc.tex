% This is LLNCS.DEM the demonstration file of
% the LaTeX macro package from Springer-Verlag
% for Lecture Notes in Computer Science,
% version 2.4 for LaTeX2e as of 16. April 2010
%
\documentclass{llncs}
%
%\usepackage{ctex}
\usepackage{CJKutf8}
\usepackage{graphicx}
\usepackage{makeidx}  % allows for indexgeneration
\usepackage{amsmath}
\begin{document}
%
%\frontmatter          % for the preliminaries
%
%\pagestyle{headings}  % switches on printing of running heads

%\addtocmark{Hamiltonian Mechanics} % additional mark in the TOC

%
%\mainmatter              % start of the contributions
%
\title{Nders at NTCIR-13 Short Text Conversation 2 Task}
%
\titlerunning{}  % abbreviated title (for running head)
%                                     also used for the TOC unless
%                                     \toctitle is used
%
\author{Liansheng Lin\inst{1} \and Han Ni\inst{2} \and Ge Xu\inst{3}}
%
%\authorrunning{Ivar Ekeland et al.} % abbreviated author list (for running head)
%
%%%% list of authors for the TOC (use if author list has to be modified)
%\tocauthor{Ivar Ekeland, Roger Temam, Jeffrey Dean, David Grove,
%Craig Chambers, Kim B. Bruce, and Elisa Bertino}
%
\institute{Fuzhou University, Fuzhou, China \\
\email{linliansheng@nd.com.cn}
\and NedDragon Websoft Inc., Fuzhou, China \\
\email{ni.han.ms6@foxmail.com}
\and Minjiang University, Fuzhou, China \\
\email{xuge@pku.edu.cn}
}

\maketitle              % typeset the title of the contribution

\begin{abstract}
This paper describes our approaches at NTCIR-13 short text conversation 2 
(STC-2) task (Chinese). For a new post, our system firstly retrieves similar posts 
in the repository and gets their corresponding comments, and then finds the 
related comments directly from the repository. Moreover, we devises a pattern-idf to rerank the candidates from above. Our best run achieved 0.4780 for mean nG@1, 0.5497 for mean P+, 
and 0.5882 for mean nERR@10, and respectively ranked 4th, 5th, 5th among 22 
teams. 
\keywords{Short text conversation, LSI, Word2vec, Pattern-idf}
\end{abstract}

\section{Introduction}
We participated in the NTCIR-13 Short Text Conversation 2 (STC-2) Chinese 
subtask. Given a new post, this task aims to retrieve an appropriate comment 
from a large post-comment repository (Retrieval-based method) or generate a new 
appropriate comment (Generation-based method). Our system chooses the 
retrieval-based method. 

The retrieved or generated comment for the new post is judged from four 
criteria: Coherent, Topically relevant, Non-repetitive and Context 
independent\cite{}. The primary criterion for a suitable comment we consider is 
topically relevant. In other words, this comment should be talking about the 
same topic with the given post. In a previous work of retrieval-based STC, 
Ji et al.\cite{} directly focus on the similarity between this new post and 
each comments. That is, when the new post and a candidate comment share same 
words or phrases, we can strongly infer that they may be on the same topic.

However, this is not a prerequisite of being a proper response of the post. 
For example, **(to do)** . In this case, post and comment is different in words 
but coherent. Such kind of comment is also non-repetitive. 

To address this problem, we train Word2Vec\cite{} model to obtain the similarity 
between new post and retrieved comment, LDA\cite() model and LSA\cite() model to obtain the 
degree of relateness. By combining them together, we proposed a simialrity 
score to search comment candidates, and achieves a good performance.

Based on a hypothesis, similar posts has similar corresponding comments, we try 
to find the similar posts to the new post and get their corresponding comments 
as a supplement for the candidates. In addition to Word2Vec model and LSA model, 
we also introduce langugae model trained by LSTM\cite() to compute post-post 
similarity.

In the last step, we rank the cadidates by Random Walk and pattern-idf 
respectively. Result shows that pattern-idf improve the performance while 
Random Walk deteriorate it instead.

The remainder of this paper is organized as follows: Section 2 describes our 
systems in detail. Our experimental results are presented in Section 3. 
We make conclusions in Section 4 .

\section{System Architecture}
The architecture of our system is described as Figure 1. It includes the following {} components.

\subsection{Preprocessing}
There are some traditional Chinese, specific symbols in raw text. We convert 
traditional Chinese to simplified Chinese and remove the specific symbols 
in order to clean the text. 

Unlike English words in a sentence are separated by spaces, Chinese short texts 
are written without any symbol between characters. So the word segmentation 
becomes necessary. We choose nlpir\footnotemark\ to segment the chinese text.

The following example in Table 3 shows the origin text, clean text and 
segentation result.

\footnotetext{http://ictclas.nlpir.org/}

\subsection{Similarity Features}
In order to compute the degree of similarity or relateness between two 
sentences, we convert text sentence into continuous vector representations 
with some techniques including LSA, LDA, Word2Vec, LSTM.

\subsubsection{LSA}
Latent semantic analysis (LSA) is a technique of analyzing relationships between a set of documents and the terms they contain by producing a set of concepts related to the documents and terms. LSA assumes that words that are close in meaning will occur in similar pieces of text (the distributional hypothesis). A matrix containing word counts per paragraph (rows represent unique words and columns represent each paragraph) is constructed from a large piece of text and a mathematical technique called singular value decomposition (SVD) is used to reduce the number of rows while preserving the similarity structure among columns. We combine each post with its corresponding comments to be a document, then we train LSA model on these documents with gensim\footnotemark\. From trained model, we can get vector for each chinese word.
Then, we can get vector representation of a sentence by Eq.1:
\begin{equation}
   V = \frac{1}{n} * \sum_1^n v_i 
\end{equation}
Here, capital $V$ refers to vector of a sentence, $v$ refers to vector of each word in the sentence, and $n$ is the length of the sentence.

\footnotetext{A python library for topic modelling, https://radimrehurek.com/gensim/index.html}

\subsubsection{LDA}
Latent Dirichlet allocation (LDA) is a generative statistical model that allows sets of observations to be explained by unobserved groups that explain why some parts of the data are similar. For example, if observations are words collected into documents, it posits that each document is a mixture of a small number of topics and that each word's creation is attributable to one of the document's topics. Like training LSA model, we combine each post with its corresponding comments to be a document, then we train LDA model on these documents with gensim. With trained LDA model, we can transform new, unseen documents into LDA topic distributions. We regard a sentence(a post, or a comment) as a document, convert it into plain bag-of-words count vector, and then index LDA model to obtain a vector representation of the sentence. 

\subsubsection{Word2Vec}
Word2Vec is an efficient tool for computing continuous distributed 
representations of words. We train our Word2Vec model on provided training 
text corpus with skip-gram architecture where window size is 10, vector 
length is 300 and min count is 20 to remove infrequent words. Vector 
representation for each chinese word is directly obtained from trained model. 
Then, we can get vector representation of a sentence by Eq.1.

\subsubsection{LSTM}
Long short-term memory (LSTM) is a recurrent neural network (RNN) architecture that remembers values over arbitrary intervals...

\subsubsection{Cosine Similarity} 
After convert sentence into vectors, we compute similarity of two sentences by cosine similarity:
\begin{equation}
   Sim(s_1, s_2) = \frac{V_1 * V_2}{\left \| V_1 \right \| \left \| V_2 \right \|}
\end{equation}
Here, $V_1$ refers to vector representation of sentence $s_1$, $V_2$ refers to vector representation of sentence $s_2$.

With sentence vectors from different models, we get corresponding sentence similarity. We use $Sim_{LSA}$ to denote sentence similarity based on LSA model, analogously, $Sim_{LDA}$ to denote sentence similarity based on LDA model and $Sim_{W2V}$ to denote sentence similarity based on Word2Vec model.

\subsection{Candidates Generation}

\subsubsection{Similar Posts}
Based on a hypothesis that similar posts has similar corresponding comments, we firstly find top-10 similar posts with a ranking score combining LSA, Word2Vec, and LSTM model:
\begin{equation}
   Score_{q,p}(q, p) = Sim_{LSA}(q, p) * Sim_{W2V}(q, p) * Sim_{LSTM}(q, p)
\end{equation}
Here, $q$ denotes the query(the new post) $p$ denote the post from repository.

Then, we get corresponding comments to the top-10 similar posts as first comment candidates, denoted as $C_1$.

\subsubsection{Comment Candidats}
Since Word2Vec model captures semantic similarity, LSA or LDA reflects topic relateness, we combine LSA or LDA with Word2Vec respectively to directly retrieve top-N appropriate comments to the new post from all comments in the repository. N is equal to the number of comment candidates $C_1$. 
\begin{equation}
   Score_{q,c}^1(q, c) = Sim_{LSA}(q, c) * Sim_{W2V}(q, c)
\end{equation}
\begin{equation}
   Score_{q,c}^2(q, c) = Sim_{LDA}(q, c) * Sim_{W2V}(q, c)
\end{equation}
Here, $q$ denotes the query(the new post) $c$ denote the comment from repository.

We combine the retrived top-N comments by Word2Vec model and LSA model with $C_1$ as final comment candidats for further reranking.

\subsection{Reranking}

\subsubsection{Random Walk}

\subsubsection{Pattern-idf}

\section{Experiments}
We submitted five runs for comparison and analysis:

\begin{enumerate}
  \item{Nders-C-R5: } Use $Sim_{w2v}*Sim_{LDA}$ as a ranking score to directly 
  retrieve and get top-10 comments from all comments.
  \item{Nders-C-R4: } Use $Sim_{w2v}*Sim_{LSI}$ as a ranking score to directly 
  retrieve and get top-10 comments from all comments.
  \item{Nders-C-R3: }
  \item{Nders-C-R2: }
  \item{Nders-C-R1: }
\end{enumerate}

\section{Conclusions}

In this paper, we  

%
% ---- Bibliography ----
%
\begin{thebibliography}{5}




\end{thebibliography}

\end{document}



